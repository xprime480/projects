\item[INT\_DATA] has 9 numeric and 0 NULL values
See Table \ref{tab:INTDATA}
\begin{table}[ht]
\caption{Statistics for INT\_DATA} \label{tab:INTDATA}
\centering
\begin{tabular}{|rrrrr|}
\hline
Quartiles & & & &  \\
1.00 & 2.00 & 3.00 & 4.00 & 6.00 \\
\hline
Average & 3.00 & & Std Dev & 1.73 \\
\hline
\end{tabular}
\end{table}
\item[STR\_DATA] has 5 distinct values and 0 NULL values.
See Table \ref{tab:STRDATA}
\begin{table}
\caption{Top 5 values for STR\_DATA} \label{tab:STRDATA}
\centering
\begin{tabular}{|r|l|}
\hline
Value & Count \\
\hline
x & 4 \\
zoo & 2 \\
y & 1 \\
cats & 1 \\
able & 1 \\
\hline
\end{tabular}
\end{table}
\item[NOT\_INT\_DATA] has 9 distinct values and 0 NULL values.
See Table \ref{tab:NOTINTDATA}
\begin{table}
\caption{Top 9 values for NOT\_INT\_DATA} \label{tab:NOTINTDATA}
\centering
\begin{tabular}{|r|l|}
\hline
Value & Count \\
\hline
zero & 1 \\
1 & 1 \\
3 & 1 \\
2 & 1 \\
5 & 1 \\
4 & 1 \\
7 & 1 \\
6 & 1 \\
8 & 1 \\
\hline
\end{tabular}
\end{table}
