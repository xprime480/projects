\documentclass[12pt]{article}
\usepackage{times}
\usepackage{graphicx}
\usepackage{float}

% Examples go in boxes
\floatstyle{boxed}
\newfloat{example}{h}{loe}
\floatname{example}{Example}

\newcommand{\sloth}{\textsc{sloth}}
\newcommand{\Sloth}{\textsc{Sloth}}
%\newcommand{\stackcmd}[3]{\textbf{#1}\,({#2} $\rightarrow{}$ {#3})}
\newcommand{\stackcmd}[3]{\item[{#1}] \hfill ({#2} $\rightarrow{}$ {#3}) \\}

% BEGIN FORMATTING
\baselineskip=18pt
\hsize=6in
%\topmargin=-0.25in
\textheight=8.5in
\evensidemargin=0.0in
\oddsidemargin=0.0in
\pagestyle{myheadings}
\markright{\sloth}

\bibliographystyle{plain}

% BEGIN MACRO DEFINITIONS

\begin{document}

\title{The \Sloth{} Programming Language}
\author{Mike Davis}
%%\date{February 1, 2007}
\maketitle

%%\tableofcontents\newpage
%%\listoffigures\newpage
%%\listoftables\newpage

\section*{Introduction}

This document describes the \sloth{} programming language.  The key
features of \sloth{} are lazy evaluation of sequences and stack based
programming. 

\section*{A Short Tour of \Sloth}

This section gives a short tour of \sloth.  

\subsection*{Literals}

This section describes the type literals which the interpreter
accepts.

\begin{description}

\stackcmd{BOOLEAN}{}{boolval} The literals '@t' and '@f' produce the
boolean values true and false, respectively.  The capitalized versions
produce the same effect.

\stackcmd{INTEGER}{}{intval} Any sequence that starts with a
decimal digit and contains only decimal digits will cause the decimal
value of the input to be pushed to the top of the stack. 

\stackcmd{SYMBOL}{}{symbol}  Any sequence beginning with a colon,
e.g. ':symbol' causes a symbol reference to be pushed onto the stack.
All symbols are global and there is only one symbol with a given name
in the system at a time.

\stackcmd{STRING}{}{string}  Any sequence beginning and ending with a
double quotation represents a string.  There is not character quoting
and the string must end before the next newline.

\stackcmd{[optional-forms $\dots$]}{}{list\-sequence}  Any sequence
surrounded by square brackets generates a list on the stack.  The
forms can be literals, generators, or other forms.

\end{description}


\subsection*{Generators}

This section describes generators, i.e., input that takes nothing from
the stack and leaves one or more objects on the stack.

\begin{description}

\stackcmd{naturals}{}{sequence} The literal 'naturals' causes a
seqeunce object to be pushed to the top of the stack.  This sequence
object generates the values from 1 to $2^{31}-1$.

\stackcmd{fibonacci}{}{sequence}  The literal 'fibonacci' causes a
sequence object to be pushed to the top of the stack.  This sequence
object generates the Fibonacci sequence 1, 1, $\dots$, $2^{31}-1$.

\end{description}

\subsection*{Stack Manipulators}

This section describes commands that take input from the stack,
possibly writing output back to the stack.

\begin{description}

\stackcmd{dup}{form}{form form}  The command 'dup' makes a copy of the
top element of the stack onto the stack.

\stackcmd{eval}{raw\-form}{result\-form}  The command 'eval' removes
the top form from the stack and evaluates it.  The result of the
evaluation is pushed back on the stack.  The result may be a message
if evaluation resulted in an error.  If the evaluation results in an
infinite sequence, the command never returns.  Literals evaluate to
themselves.  Sequences evalute to a non-lazy evaluations of their
contents, in order.  Therefore, if an infinite sequence,
e.g. 'naturals', is evaluated, evaluation will never terminate.

\stackcmd{pop}{form}{} The command 'pop' removes the top element from
the stack.

\stackcmd{print}{form}{} The command 'print' removes the top element
from the stack and displays a printable representation to the standard
output.

\stackcmd{printall}{forms $\dots$}{} The command 'printall' removes
all elements from the stack and displays a printable representation of
each to the standard output.  The top of stack is printed first.

\stackcmd{swap}{form1 form2}{form2 form1} The command 'swap removes
two elements from the stack and and puts them back onto the stack in
the opposite order.

\end{description}

\subsection*{Operations}

This section describes operations that take 1 or more arguments from
the stack and produce a new lazy form that is pushed on the stack.
In any case, if the described inputs are incorrect, then a message
form is left on the stack.  (The interpreter will print out these
messages.)


\begin{description}

\stackcmd{append}{sequence\-1 sequence\-2}{sequence}  The 'append'
command takes two sequences from the stack and pushes onto the stack a
sequence that is all of the items from sequence\-1 followed by all the
items from sequence\-2. 

\stackcmd{count}{sequence}{intval}  The 'count' command takes a
sequence off the stack and pushes onto the stack a lazy integer value
representing the number of top level forms in the sequence.

\stackcmd{drop}{sequence intval}{sequence}  The 'drop' command takes a
sequence and an integer from the stack and pushes onto the stack a
sequence that is the input sequence with the first intval items
removed.  If the intval is negative, zero items are removed.

\stackcmd{flatten}{sequence}{sequence}  The 'flatten' command takes a
sequence from the stack and pushes a sequence that contains all the
leaf items from the original sequence as a single sequence.  For
example, if the input is [[1 2 3] [4 5 6]], then the output is [1 2 3
4 5 6].

\stackcmd{gfibonacci}{intval intval}{sequence}  The 'gfibonacci'
command takes two integer values off the stack and pushes a lazy
sequence that is the ``generalized'' Fibonacci sequence seeded by
those two values.  The first two values are the inputs, and each value
after that is the sum of its two prececessors.  For example, if the
seeds are $a$ and $b$, the values are $a, b, a+b, a+2b, 2a+3b, 3a+5b,
\dots$.  The normal Fibonacci sequence is seeded by $1, 1$.

\stackcmd{minmax}{sequence}{sequence}  The 'minmax' command takes a
sequence off the stack and pushes a lazy sequence onto the stack.  If
the input sequence is empty, the output is empty.  Otherwise, the
output is a sequence of length two containing the minimum value in the
input sequence and the maximum value in the input sequence.

\stackcmd{repeat}{form}{sequence}  The 'repeat' command takes a form
off the stack and returns a lazy sequence that returns form forever.

\stackcmd{set}{form symbol}{form}  The 'set' command takes a form and
a symbol off the stack, sets the value of the symbol to form, and
pushes the form back onto the stack.

\stackcmd{take}{sequence intval}{sequence}  The 'take' command takes a
sequence and an integer value off the stack and pushes onto the stack
a lazy sequence consisting of the first intval values from the
sequence.

\stackcmd{takeif}{boolean-sequence sequence}{sequence}  The 'takeif
command takes two sequences off the stack and pushes onto the stack a
lazy sequence representing consisting of items from the second
sequence where the corresponding position in the first sequence is
@t.

\end{description}

\subsection*{Operators}

This section describes mathematical and logical operators.  Unary
operators take a single form from the stack, while binary
operators can take two.

For a binary operator, if the inputs are a pair of sequences, the
output is a lazy sequence where $Out_i = In_{1,i} \odot In_{2,i}$,
where $\odot$ is the documented operation, $Out$ is the output
sequence and $In_{1}$ and $In_{2}$ are the first and second output
sequences. $i$ ranges up to the length of the shorter input sequence.
If the inputs are both non-sequences, the output is a non-sequence
also.  If one input is a non-sequence and the other a sequence,
evaluation occurs as if the non-sequence were a lazy sequence
containing an infinite number of copies of the input.

For a unary argument, an input that is a sequence produces an output
that is also a sequence.  An input that is not a sequence produces an
output that is not a sequence.



\begin{description}

\stackcmd{$+$}{form1 form2}{form} The '$+$' operator takes two inputs
and returns their sum.  It only works on integers and sequences of
integers.

\stackcmd{$-$}{form1 form2}{form} The '$-$' operator takes two inputs
and returns their difference, form1 $-$ form2.  It only works on
integers and sequences of integers.

\stackcmd{$*$}{form1 form2}{form} The '$*$' operator takes two inputs
and returns their product, form1 $*$ form2.  It only works on integers
and sequences of integers.

\stackcmd{$/$}{form1 form2}{form} The '$/$' operator takes two inputs
and returns their product, form1 $/$ form2.  It only works on integers
and sequences of integers.  An error form is returned if form2 is zero.

\stackcmd{\%}{form1 form2}{form} The '\%' operator takes two inputs
and returns their product, form1 \% form2.  It only works on integers
and sequences of integers.  An error form is returned if form2 is zero.

\end{description}

\end{document}
